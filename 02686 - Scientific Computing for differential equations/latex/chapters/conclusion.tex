\chapter{Conclusion}
We have now investigated the Van der Pol problems using different methods (and CSTR using DoPri54), and have a better understanding of which methods work better for different settings. For the Van der Pol problem, the choice of solution method becomes critical when we have large $\mu$-values. In these cases, the problem has stiff regions of rapid and large changes in the dynamics, making approximating the real solution difficult. The explicit adaptive Euler method was relatively slow due to many rejected states and low step size when better accuracy was required. It did not handle the Van der Pol problem very well. The implicit method had better convergence properties, but eventually required many more function evaluations, making it relatively slow. The explicit RK4 adaptive method was just as fast as the Matlab ODE solvers. It had a lower number of function evaluations than the implicit Euler. The DOPRI54 method behaves very much like the RK4, but is faster for the more stiff problem with $\mu = 20$. The ESDIRK23 method has almost no rejected states, but also requires more function evaluations. 

In the SDE problem, we saw  how the discrete points in the Van der Pol problem became extremely unpredictable for the second drift function (state dependent). For the independent state diffusion, the discrete points were more noisy, but a solution was better obtainable. This was the case for both explicit-explicit and implicit-explicit SDE methods. Again in the CSRT problem, the implicit-explicit method was more noisy in.

As a general conclusion, we have now looked at many different numerical algorithms, and have identified the pros and cons of each method. Every method has its own advantages and disadvantages, and the goal and budget for the implementation of the given problem is the "limiting factor". 

The key takeaway most be that whenever you need to use a scientific method for solving a differential equation it is really important to consider how your problem behaves and what methods is suitable. Generally speaking, you want to maximize you "bang-for-buck", such that you achieve the highest possible accuracy in the easiest way possible. For stiff problems, this requires good convergence properties of your method, hence you would prefer some implicit method. Otherwise, you would use a simple method that can give you your desired accuracy. If you need a good accuracy, it is therefore natural to select a higher order method. To maintain speed, you often want some kind of step control. This is why the standard choice of method is often the DoPri54, as is the case with ODE45 from Matlab.

All code in this report has been created in collaboration with Andreas Engly, s170303, and Anton Larsen, s174356. All code used for generation of plots etc is only given in the zip file handed in electronically. 

